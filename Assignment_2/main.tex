\documentclass[journal,12pt,twocolumn]{IEEEtran}

\usepackage{setspace}
\usepackage{gensymb}
\singlespacing
\usepackage[cmex10]{amsmath}

\usepackage{amsthm}

\usepackage{mathrsfs}
\usepackage{txfonts}
\usepackage{stfloats}
\usepackage{bm}
\usepackage{cite}
\usepackage{cases}
\usepackage{subfig}

\usepackage{longtable}
\usepackage{multirow}

\usepackage{enumitem}
\usepackage{mathtools}
\usepackage{steinmetz}
\usepackage{tikz}
\usepackage{circuitikz}
\usepackage{verbatim}
\usepackage{tfrupee}
\usepackage[breaklinks=true]{hyperref}
\usepackage{graphicx}
\usepackage{tkz-euclide}

\usetikzlibrary{calc,math}
\usepackage{listings}
    \usepackage{color}                                            %%
    \usepackage{array}                                            %%
    \usepackage{longtable}                                        %%
    \usepackage{calc}                                             %%
    \usepackage{multirow}                                         %%
    \usepackage{hhline}                                           %%
    \usepackage{ifthen}                                           %%
    \usepackage{lscape}     
\usepackage{multicol}
\usepackage{chngcntr}

\DeclareMathOperator*{\Res}{Res}

\renewcommand\thesection{\arabic{section}}
\renewcommand\thesubsection{\thesection.\arabic{subsection}}
\renewcommand\thesubsubsection{\thesubsection.\arabic{subsubsection}}

\renewcommand\thesectiondis{\arabic{section}}
\renewcommand\thesubsectiondis{\thesectiondis.\arabic{subsection}}
\renewcommand\thesubsubsectiondis{\thesubsectiondis.\arabic{subsubsection}}


\hyphenation{op-tical net-works semi-conduc-tor}
\def\inputGnumericTable{}                                 %%

\lstset{
%language=C,
frame=single, 
breaklines=true,
columns=fullflexible
}
\begin{document}

\newcommand{\BEQA}{\begin{eqnarray}}
\newcommand{\EEQA}{\end{eqnarray}}
\newcommand{\define}{\stackrel{\triangle}{=}}
\bibliographystyle{IEEEtran}
\raggedbottom
\setlength{\parindent}{0pt}
\providecommand{\mbf}{\mathbf}
\providecommand{\pr}[1]{\ensuremath{\Pr\left(#1\right)}}
\providecommand{\qfunc}[1]{\ensuremath{Q\left(#1\right)}}
\providecommand{\sbrak}[1]{\ensuremath{{}\left[#1\right]}}
\providecommand{\lsbrak}[1]{\ensuremath{{}\left[#1\right.}}
\providecommand{\rsbrak}[1]{\ensuremath{{}\left.#1\right]}}
\providecommand{\brak}[1]{\ensuremath{\left(#1\right)}}
\providecommand{\lbrak}[1]{\ensuremath{\left(#1\right.}}
\providecommand{\rbrak}[1]{\ensuremath{\left.#1\right)}}
\providecommand{\cbrak}[1]{\ensuremath{\left\{#1\right\}}}
\providecommand{\lcbrak}[1]{\ensuremath{\left\{#1\right.}}
\providecommand{\rcbrak}[1]{\ensuremath{\left.#1\right\}}}
\theoremstyle{remark}
\newtheorem{rem}{Remark}
\newcommand{\sgn}{\mathop{\mathrm{sgn}}}
\providecommand{\abs}[1]{\vert#1\vert}
\providecommand{\res}[1]{\Res\displaylimits_{#1}} 
\providecommand{\norm}[1]{\lVert#1\rVert}
%\providecommand{\norm}[1]{\lVert#1\rVert}
\providecommand{\mtx}[1]{\mathbf{#1}}
\providecommand{\mean}[1]{E[ #1 ]}
\providecommand{\fourier}{\overset{\mathcal{F}}{ \rightleftharpoons}}
%\providecommand{\hilbert}{\overset{\mathcal{H}}{ \rightleftharpoons}}
\providecommand{\system}{\overset{\mathcal{H}}{ \longleftrightarrow}}
	%\newcommand{\solution}[2]{\textbf{Solution:}{#1}}
\newcommand{\solution}{\noindent \textbf{Solution: }}
\newcommand{\cosec}{\,\text{cosec}\,}
\providecommand{\dec}[2]{\ensuremath{\overset{#1}{\underset{#2}{\gtrless}}}}
\newcommand{\myvec}[1]{\ensuremath{\begin{pmatrix}#1\end{pmatrix}}}
\newcommand{\mydet}[1]{\ensuremath{\begin{vmatrix}#1\end{vmatrix}}}
\numberwithin{equation}{subsection}
\makeatletter
\@addtoreset{figure}{problem}
\makeatother
\let\StandardTheFigure\thefigure
\let\vec\mathbf
\renewcommand{\thefigure}{\theproblem}
\def\putbox#1#2#3{\makebox[0in][l]{\makebox[#1][l]{}\raisebox{\baselineskip}[0in][0in]{\raisebox{#2}[0in][0in]{#3}}}}
     \def\rightbox#1{\makebox[0in][r]{#1}}
     \def\centbox#1{\makebox[0in]{#1}}
     \def\topbox#1{\raisebox{-\baselineskip}[0in][0in]{#1}}
     \def\midbox#1{\raisebox{-0.5\baselineskip}[0in][0in]{#1}}
\vspace{3cm}
\title{Assignment 1}
\author{Dishank Jain - AI20BTECH11011}
\maketitle
\newpage
\bigskip
\renewcommand{\thefigure}{\theenumi}
\renewcommand{\thetable}{\theenumi}
Download all python codes from 
\begin{lstlisting}
https://github.com/Dishank422/AI1103-Probability-and-random-variables/blob/main/Assignment_2/codes
\end{lstlisting}
%
and latex-tikz codes from 
%
\begin{lstlisting}
https://github.com/Dishank422/AI1103-Probability-and-random-variables/blob/main/Assignment_2/main.tex
\end{lstlisting}
\section{Problem}
(Gate 11)The probability that a given positive number lying between 1 and 100 (both inclusive) is NOT divisible by 2, 3 or 5 is...... 
\section{Solution}
Let $X \in  \{1,2,...,100\}$ be the random variable representing the outcome for random selection of a number in $\{1,...,100\}$.

Since X has a uniform distribution, the probability mass function (pmf) is represented as 
\begin{align}
    \pr{X  = n} = 
    \begin{cases}
        \frac{1}{100} & 1  \le n \le 100 \\
        0 & otherwise
    \end{cases}
\end{align}
Let A represent the event that the number is divisible by 2.
Let B represent the event that the number is divisible by 3.
Let C represent the event that the number is divisible by 5.

We need to find the probability that the number is not divisible by 2, 3 or 5. Thus we need to find $1 - \pr{A+B+C}$

We know 
\begin{multline}
    \pr{A+B+C} = \pr{A} + \pr{B} + \pr{C} \\- \pr{AB} - \pr{BC} \\- \pr{AC} + \pr{ABC}\label{Shurururururururu}    
\end{multline}

\begin{center}
\begin{tabular}{|c|c|c|}
\hline
Event & Interpretation  & Probability  \\
\hline
A    & n is divisible by 2  & $\cfrac{50}{100}$ \\
B    & n is divisible by 3  & $\cfrac{33}{100}$ \\
C    & n is divisible by 5  & $\cfrac{20}{100}$ \\
AB   & n is divisible by 6  & $\cfrac{16}{100}$ \\
BC   & n is divisible by 15 & $\cfrac{6}{100}$ \\
AC   & n is divisible by 10 & $\cfrac{10}{100}$ \\
ABC  & n is divisible by 30 & $\cfrac{3}{100}$ \\
\hline
\end{tabular}
\end{center}

Substituting in \eqref{Shurururururururu}, we get
\begin{multline}
    \pr{A+B+C} = \frac{50}{100} + \frac{33}{100} + \frac{20}{100} \\- \frac{16}{100} - \frac{6}{100} - \frac{10}{100} + \frac{3}{100} 
\end{multline}
Thus, 
\begin{align}
    \pr{A+B+C} = \frac{74}{100}
\end{align}
Thus required probability = 
\begin{align}
    1 - \pr{A+B+C} = \frac{26}{100}
\end{align}
\end{document}
