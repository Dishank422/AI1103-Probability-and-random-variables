\documentclass[journal,12pt,twocolumn]{IEEEtran}

\usepackage{setspace}
\usepackage{gensymb}
\singlespacing
\usepackage[cmex10]{amsmath}

\usepackage{amsthm}

\usepackage{mathrsfs}
\usepackage{txfonts}
\usepackage{stfloats}
\usepackage{bm}
\usepackage{cite}
\usepackage{cases}
\usepackage{subfig}

\usepackage{longtable}
\usepackage{multirow}

\usepackage{enumitem}
\usepackage{mathtools}
\usepackage{steinmetz}
\usepackage{tikz}
\usepackage{circuitikz}
\usepackage{verbatim}
\usepackage{tfrupee}
\usepackage[breaklinks=true]{hyperref}
\usepackage{graphicx}
\usepackage{tkz-euclide}

\usetikzlibrary{calc,math}
\usepackage{listings}
    \usepackage{color}                                            %%
    \usepackage{array}                                            %%
    \usepackage{longtable}                                        %%
    \usepackage{calc}                                             %%
    \usepackage{multirow}                                         %%
    \usepackage{hhline}                                           %%
    \usepackage{ifthen}                                           %%
    \usepackage{lscape}     
\usepackage{multicol}
\usepackage{chngcntr}

\DeclareMathOperator*{\Res}{Res}

\renewcommand\thesection{\arabic{section}}
\renewcommand\thesubsection{\thesection.\arabic{subsection}}
\renewcommand\thesubsubsection{\thesubsection.\arabic{subsubsection}}

\renewcommand\thesectiondis{\arabic{section}}
\renewcommand\thesubsectiondis{\thesectiondis.\arabic{subsection}}
\renewcommand\thesubsubsectiondis{\thesubsectiondis.\arabic{subsubsection}}


\hyphenation{op-tical net-works semi-conduc-tor}
\def\inputGnumericTable{}                                 %%

\lstset{
%language=C,
frame=single, 
breaklines=true,
columns=fullflexible
}
\begin{document}

\newcommand{\BEQA}{\begin{eqnarray}}
\newcommand{\EEQA}{\end{eqnarray}}
\newcommand{\define}{\stackrel{\triangle}{=}}
\bibliographystyle{IEEEtran}
\raggedbottom
\setlength{\parindent}{0pt}
\providecommand{\mbf}{\mathbf}
\providecommand{\pr}[1]{\ensuremath{\Pr\left(#1\right)}}
\providecommand{\qfunc}[1]{\ensuremath{Q\left(#1\right)}}
\providecommand{\sbrak}[1]{\ensuremath{{}\left[#1\right]}}
\providecommand{\lsbrak}[1]{\ensuremath{{}\left[#1\right.}}
\providecommand{\rsbrak}[1]{\ensuremath{{}\left.#1\right]}}
\providecommand{\brak}[1]{\ensuremath{\left(#1\right)}}
\providecommand{\lbrak}[1]{\ensuremath{\left(#1\right.}}
\providecommand{\rbrak}[1]{\ensuremath{\left.#1\right)}}
\providecommand{\cbrak}[1]{\ensuremath{\left\{#1\right\}}}
\providecommand{\lcbrak}[1]{\ensuremath{\left\{#1\right.}}
\providecommand{\rcbrak}[1]{\ensuremath{\left.#1\right\}}}
\theoremstyle{remark}
\newtheorem{rem}{Remark}
\newcommand{\sgn}{\mathop{\mathrm{sgn}}}
\providecommand{\abs}[1]{\vert#1\vert}
\providecommand{\res}[1]{\Res\displaylimits_{#1}} 
\providecommand{\norm}[1]{\lVert#1\rVert}
%\providecommand{\norm}[1]{\lVert#1\rVert}
\providecommand{\mtx}[1]{\mathbf{#1}}
\providecommand{\mean}[1]{E[ #1 ]}
\providecommand{\fourier}{\overset{\mathcal{F}}{ \rightleftharpoons}}
%\providecommand{\hilbert}{\overset{\mathcal{H}}{ \rightleftharpoons}}
\providecommand{\system}{\overset{\mathcal{H}}{ \longleftrightarrow}}
	%\newcommand{\solution}[2]{\textbf{Solution:}{#1}}
\newcommand{\solution}{\noindent \textbf{Solution: }}
\newcommand{\cosec}{\,\text{cosec}\,}
\providecommand{\dec}[2]{\ensuremath{\overset{#1}{\underset{#2}{\gtrless}}}}
\newcommand{\myvec}[1]{\ensuremath{\begin{pmatrix}#1\end{pmatrix}}}
\newcommand{\mydet}[1]{\ensuremath{\begin{vmatrix}#1\end{vmatrix}}}
\numberwithin{equation}{subsection}
\makeatletter
\@addtoreset{figure}{problem}
\makeatother
\let\StandardTheFigure\thefigure
\let\vec\mathbf
\renewcommand{\thefigure}{\theproblem}
\def\putbox#1#2#3{\makebox[0in][l]{\makebox[#1][l]{}\raisebox{\baselineskip}[0in][0in]{\raisebox{#2}[0in][0in]{#3}}}}
     \def\rightbox#1{\makebox[0in][r]{#1}}
     \def\centbox#1{\makebox[0in]{#1}}
     \def\topbox#1{\raisebox{-\baselineskip}[0in][0in]{#1}}
     \def\midbox#1{\raisebox{-0.5\baselineskip}[0in][0in]{#1}}
\vspace{3cm}
\title{Assignment 8}
\author{Dishank Jain - AI20BTECH11011}
\maketitle
\newpage
\bigskip
\renewcommand{\thefigure}{\theenumi}
\renewcommand{\thetable}{\theenumi}
Download latex-tikz codes from 
%
\begin{lstlisting}
https://github.com/Dishank422/AI1103-Probability-and-random-variables/blob/main/Assignment_8/main.tex
\end{lstlisting}
\section{Problem}
(CSIR-UGC-NET-DEC 2015, Q. 108) Suppose that $(X,Y)$ has a joint probability distribution with the marginal distribution of $X$ being N(0,1) and $E(Y|X=x)=x^3$ for all $x \in R$. Then, which of the following statements are true?
\begin{enumerate}
    \item Corr$(X,Y) = 0$
    \item Corr$(X,Y) > 0$
    \item Corr$(X,Y) < 0$
    \item X and Y are independent
\end{enumerate}
\section{Solution}
The following result shall be useful later. For $n \in N$
\begin{align}
    \int_{-\infty}^{\infty} \dfrac{x^n e^{\frac{-x^2}{2}}}{\sqrt{2\pi}}dx = 
    \begin{cases}
    0 & n\; is\; odd\\
    (n-1)\times...\times3\times1 & n\; is\; even
    \end{cases}
\end{align}
The proof for the above can be found at the end of the solution.
\begin{align}
    Corr(X,Y) = \cfrac{\sigma_{XY}^2}{\sigma_X\sigma_Y}
    \label{correlationofxy}
\end{align}

We know $X \sim N(0,1)$. Thus,
\begin{align}
    f_X(x) &= \dfrac{e^{\frac{-x^2}{2}}}{\sqrt{2\pi}}\\
    E(X) &= 0\\
    \sigma_X^2 &= 1
\end{align}
\begin{align}
    \sigma_Y^2 = E(Y^2) - E(Y)^2
    \label{varianceofy}
\end{align}
\begin{align}
    E(Y) &= \int_{-\infty}^{\infty}E(Y|X=x)f_X(x)dx\\
         &= \int_{-\infty}^{\infty}\dfrac{x^3 e^{\frac{-x^2}{2}}}{\sqrt{2\pi}}dx\\
         &= 0
\end{align}
\begin{align}
    E(Y^2) &= \int_{-\infty}^{\infty}E(Y^2|X=x)f_X(x)dx\\
           &= \int_{-\infty}^{\infty}\dfrac{x^6 e^{\frac{-x^2}{2}}}{\sqrt{2\pi}}dx\\
           &= 15
\end{align}
Substituting in \eqref{varianceofy}
\begin{align}
    \sigma_Y^2 = 15
\end{align}
\begin{align}
    \sigma_{XY}^2 = E(XY) - E(X)E(Y)
    \label{covarianceofxy}
\end{align}
\begin{align}
    E(XY) &= \int_{-\infty}^{\infty}E(XY|X=x)f_X(x)dx\\
          &= \int_{-\infty}^{\infty}\dfrac{x^4 e^{\frac{-x^2}{2}}}{\sqrt{2\pi}}dx\\
          &= 3
\end{align}
Substituting in \eqref{covarianceofxy}
\begin{align}
    \sigma_{XY}^2 = 3
\end{align}
Substituting in \eqref{correlationofxy}
\begin{align}
    Corr(X,Y) = \cfrac{3}{\sqrt{15}} > 0
\end{align}
Since $Corr(X,Y) \ne 0$, X and Y are dependent. Thus option 2 is the only correct option.


\textbf{Proof for the integral:}

If n is odd, $\dfrac{x^n e^{\frac{-x^2}{2}}}{\sqrt{2\pi}}$ is an odd function, thus
\begin{align}
    \int_{-\infty}^{\infty} \dfrac{x^n e^{\frac{-x^2}{2}}}{\sqrt{2\pi}}dx = 0
\end{align}
If n is even, 
\begin{align}
    \int_{-\infty}^{\infty} \dfrac{x^n e^{\frac{-x^2}{2}}}{\sqrt{2\pi}}dx = \int_{-\infty}^{\infty} (x^{n-1}) (\dfrac{x e^{\frac{-x^2}{2}}}{\sqrt{2\pi}})dx
\end{align}
Using integration by parts,
\begin{multline}
    \int_{-\infty}^{\infty} \dfrac{x^n e^{\frac{-x^2}{2}}}{\sqrt{2\pi}}dx = \left(x^{n-1}\int \dfrac{x e^{\frac{-x^2}{2}}}{\sqrt{2\pi}}dx\right)\biggr \vert_{-\infty}^{\infty}\\
       - (n-1)\int_{-\infty}^{\infty}x^{n-2}\left(\int \dfrac{x e^{\frac{-x^2}{2}}}{\sqrt{2\pi}}dx\right) dx
\end{multline}
\begin{align}
    &= \left(x^{n-1}(-\dfrac{e^{\frac{-x^2}{2}}}{\sqrt{2\pi}})\right)\biggr \vert_{-\infty}^{\infty}
           - (n-1)\int_{-\infty}^{\infty}x^{n-2}(-\dfrac{e^{\frac{-x^2}{2}}}{\sqrt{2\pi}}) dx\\
    &= (n-1)\int_{-\infty}^{\infty} \dfrac{x^{n-2} e^{\frac{-x^2}{2}}}{\sqrt{2\pi}}dx\\
    &= (n-1)(n-3)\int_{-\infty}^{\infty} \dfrac{x^{n-4} e^{\frac{-x^2}{2}}}{\sqrt{2\pi}}dx\\
    &= (n-1)\times...\times3\times1\int_{-\infty}^{\infty} \dfrac{x^0 e^{\frac{-x^2}{2}}}{\sqrt{2\pi}}dx\\
    &= (n-1)\times...\times3\times1
\end{align}

\end{document}
